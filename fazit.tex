\section{Fazit}
\label{sec:Fazit}
Im Rahmen der vorhandenen Zeit wurden alle beschriebenen Use Cases umgesetzt. Es können Sensorwerte empfangen, gespeichert und angezeigt werden. Die optische wie auch akustische Darstellung von kritischen Werten ist umgesetzt. Einfache Zusammenstellungen von vergangenen Werten sind möglich.\\
Die Aufgabe ist somit realisiert. Im Folgenden werden zum einen die Probleme, welche bei der Bearbeitung auftraten, kurz erläutert und zum anderen noch ein Ausblick gegeben, wie die Applikation verbessert und erweitert werden könnte.


\subsection{Probleme}
\label{subsec:Probleme}
Während der Bearbeitung des Projektes kam es zu diversen Problemen, die sich aber hauptsächlich auf den Funktionsumfang und den Implementierungsstand des sich noch in der Entwicklung befindenden eXist konzentrieren. Um Websockets innerhalb von eXist nutzen zu können, war zunächst angedacht, das Package XQWebsocketModule\footnote[1]{https://github.com/wstarcev/XQWebsocketModule} zu verwenden, da dies eine Implementierung der Sockets als ein Javaprogramm und die Verwendung innerhalb von XQuery versprach. Jedoch konnte für das Package keine gültige Archivdatei erstellt werden, welche von eXist gelesen und importiert werden kann. Ein möglicher Grund dafür könnte sein, dass dieses Modul nicht für die aktuelle Version 2.2 entwickelt wurde. Laut Github, wo das Projekt gehostet wurde, gab es seit rund zwei Jahren keine größeren Änderungen mehr. Deswegen wurde letztendlich entschieden, das Modul nicht zu verwenden, sondern die Kommunikation per Websockets auf einen Node.JS-Server auszulagern.
\\
\\
Weiterhin gab es bei der Implementierung der Queries mittels XQuery diverse Probleme bei der Benutzung bestimmter Funktionen. Beispielsweise war es mit der Funktion \textit{file:exists}, nicht möglich zu bestimmen, ob eine Datei bereits existiert oder nicht. Im konkreten Fall gab die Funktion immer \textit{False} zurück, obwohl die Datei vorhanden war. Vermutlich lässt sich dieses Problem wieder auf die nicht abgeschlossene Entwicklung von eXist zurückführen, sodass noch nicht alle von der W3C\footnote[2]{World Wide Web Consortium} definierten Funktionen in XQuery 3.0 implementiert zu sein scheinen. Gelöst wurde dieses Problem durch die Verwendung der Funktion \textit{doc-available} aus der XQuery Version 1.0, die ein zufriedenstellendes Ergebnis lieferte.

\subsection{Ausblick}
\label{subsec:Ausblick}
Das Ergebnis dieser Arbeit ist eine funktionierende Applikation, aber es gibt eine Reihe von Möglichkeiten, diese zu erweitern. Einige davon sollen hier aufgeführt werden.

\begin{itemize}
	\item \textbf{weitere Reportmöglichkeiten:} \\
	Die vorhandene Darstellung von Reporten ist sehr einfach und nicht anpassbar. Es könnten Optionen zur individuellen Formatierung oder selektiven Darstellung von Werten eingefügt werden um die Filterung von interessanten Werten zu vereinfachen.
	\item \textbf{grafische Darstellung vergangener Werte:} \\
	Zurzeit werden nur die jeweils aktuellen Werte als Text angezeigt. Optisch ansprechender und informativer wäre z.B. eine Darstellung der Werte in Diagrammen. In Kombination mit den erweiterten Reportmöglichkeiten könnten diese Diagramme auch in die Reports übernommen werden. Ein Auswahl der Diagrammart wäre ebenfalls sinnvoll.
	\item \textbf{Reaktionen auf besondere Ereignisse:} \\
	Da bei den hier beobachteten kritischen Vitalwerten eine schnelle Reaktionszeit oft ein wichtiger Faktor ist, wäre das Hinzufügen von automatischen Aktionen bei gewissen Ereignissen, z.B. kritischen Werten, sehr praktisch. Die Art dieser Aktionen kann dabei z.B. die Anpassung von Gerätefunktionen oder das schlichte Senden einer Email sein.
	\item \textbf{Echtzeittest:} \\
	Da die im vorherigen Punkt beschriebene Reaktionszeit sehr wichtig ist, muss sie auch getestet werden. Dies war im Rahmen dieser Arbeit jedoch nicht möglich. 
	\item \textbf{dynamische Anpassung an weitere Sensoren:} \\
	Die offene Gestaltung der XML zur Übertragung der Werte erlaubt das Hinzufügen von weiteren Sensortypen. Diese müssen jedoch auch dargestellt werden, was zurzeit nicht möglich ist, deswegen wäre eine dynamische Anpassung der GUI an weitere Sensoren nötig. 
	\item \textbf{Auswahl bestimmter Sensoren:} \\
	Vor allem im Zusammenhang mit dem vorigen Punkt kann es nützlich werden, wenn man nur bestimmte Sensorwerte angezeigt bekommen möchte.
	\item \textbf{Sicherheit:} \\
	Vitalwerte sind personenbezogene Daten und müssen geschützt werden. Dazu sollte es eine Funktion zur Benutzersteuerung geben. Ebenso könnte die Verbindung an sich gesichert werden.
\end{itemize}

