\section{Implementierung}
\label{sec:Implementierung}
Das folgende Kapitel beschreibt beispielhaft die Umsetzung des im vorigen Abschnitt beschriebenen Entwurfs. Dabei wird auf einzelne Programmabschnitte näher eingegangen.

\subsection{Node.JS-Server}
\label{subsec:NodeJSServer}
Der Node.JS-Server besteht aus zwei Instanzen. Eine Instanz ist als ein REST-Service implementiert. Sie stellt eine POST-Ressource zur Verfügung, über die die Daten der Sensoren an das eXist geschickt werden. Hierfür gibt es eine \textit{route}-Funktion, die ein Request entgegennimmt und prüft, welche URL angesprochen wird. 

\begin{lstlisting}
function route(handle, pathname, response, request) {
    if(typeof handle[pathname] === 'function'){
        handle[pathname](response, request);
    } else {
        console.log("No request handler found for " + pathname);
        response.writeHead(404, {"Content-Type": "text/plain"});
        response.write("404 Not found");
        response.end();
    }
}
\end{lstlisting}

Node.JS extrahiert den \textit{pathname}, also den Teil der URL, der nach dem Host und der Portnummer folgt, automatisch. Sollte die URL nicht zu einer Funktion zuzuordnen sein, wird ein 404-Fehler zurückgegeben mit dem Hinweis, dass die Ressource nicht gefunden wurde.

\begin{lstlisting}
handle["/postVitalwert"] = vitalWerteHandler.postVitalwert;
\end{lstlisting}

Wenn aber die URL wie im obigen Beispiel \textit{postVitalwert} ist, wird die entsprechende Funktion \textit{postVitalwert} ausgeführt. Diese Funktion sendet die empfangenen Daten an eXist weiter. Um REST-Calls in Node.JS durchzuführen, wurde das Package \textit{node-rest-client} zusätzlich installiert. Ein POST kann wie folgt implementiert werden.

\begin{lstlisting}
client.post("http://localhost:8080/exist/restxq/postSensorData", args, function(data,response) {
    var string = data.message;
});
\end{lstlisting}

Dadurch wird ein POST an die angegebene Adresse mit den spezifischen Daten in \textit{args} ausgeführt. Die Antwort wird in der übergebenen Callback-Funktion verarbeitet. Außerdem  werden die Daten von den Sensoren an die zweite Instanz des Node.JS-Servers weitergeleitet.
\\
\\
Die zweite Instanz ist für die Kommunikation zwischen den Clients und dem Server verantwortlich. Diese wird durch den Einsatz von Websockets realisiert. Das Package \textit{socket.io} ermöglicht das Erzeugen einer Connection sowie das Senden und Empfangen von Events. Sobald sich ein Client über den Socket angemeldet hat, kann auf solche Events reagiert werden.

\begin{lstlisting}
socket.on("setPersonId", function(id){
            ...
});
\end{lstlisting}

In diesem Beispiel wird beim empfangenen Event \textit{setPersonId} die Callback-Funktion mit der empfangenen ID ausgeführt. Es können aber auch Events an die angemeldeten Clients mittels der Funktion \textit{emit} geschickt werden.

\begin{lstlisting}
receiver.emit("receiveVitalWert", xml);
\end{lstlisting}

Die beiden Instanzen wurden jeweils an zwei verschiedene Ports gebunden, um z.B. zu vermeiden, dass ein Websocket-Event an den Router des REST-Services geschickt wird. Deshalb wurde für den REST-Service der Port 8888 und für den Websocket-Service der Port 8887 festgelegt.

\subsection{eXist-Server}
\label{subsec:eXistServer}
Um eXist von außerhalb ansprechen zu können, wurde der in eXist implementierte Standard \textit{RESTXQ} verwendet. Dieser ermöglicht mittels Annotationen das Bereitstellen von \textit{resource functions}, also Ressourcen, die über HTTP verfügbar sind. Es können XQuery-Funktionen über solche HTTP-Anfragen ausgeführt und die Ergebnisse zurückgegeben werden. POST-Methoden werden beispielsweise mit der Annotation \textit{\%rest:POST("\{\$body\}")} deklariert. Mit dem \textit{\$body} Element kann außerdem festgelegt werden, dass mit dem POST Daten gesendet werden sollen.

\begin{lstlisting}
declare
    %rest:POST("{$body}")
    %rest:consumes("application/xml")
    %rest:path("/postSensorData")
function ex:postSensorData($body){
    ...
};
\end{lstlisting}

Außerdem ist es möglich, eine Antwort an den anfragenden Client innerhalb von XQuery zu erstellen. Das Element \textit{$<$rest:response$>$} erzeugt mit dem Kindelement \textit{$<$http:response$>$} eine HTTP Statusmeldung mit optional angehangenen Daten. Im folgenden Beispiel wird nach erfolgreichem Speichern der Vitalwerte ein Statuscode 200 gesendet mit dem kurzen XML-Dokument \textit{$<$message$>$Document updated$<$/message$>$}

\begin{lstlisting}
<rest:response>
     <http:response status="200" message="ok">
     <http:header name="Content-Type" value="application/xml"/>
     <http:header name="Access-Control-Allow-Origin" value="*"/>
</http:response>
</rest:response>,<message>Document updated</message>
\end{lstlisting}

Für die Erstellung der Reports wurde XSLT\footnote[1]{XSL Transformation} und XSL-FO\footnote[2]{Extensible Stylesheet Language – Formatting Objects} verwendet. 

\begin{lstlisting}
let $table-fo := transform:transform($tableBody,doc("/db/apps/aal-server/test-xsl.xsl"),())
let $fo := <fo:root xmlns:fo="http://www.w3.org/1999/XSL/Format">
    <fo:layout-master-set>
        <fo:simple-page-master master-name="my-page">
            <fo:region-body margin="1in"/>
        </fo:simple-page-master>
    </fo:layout-master-set>
    <fo:page-sequence master-reference="my-page">
        <fo:flow flow-name="xsl-region-body">
            <fo:block>
                {$table-fo}
            </fo:block>
        </fo:flow>
    </fo:page-sequence>
</fo:root>
let $pdf := xslfo:render($fo, "application/pdf", ())
\end{lstlisting}

In der ersten Zeile wird das XML-Dokument, welches transformiert werden soll sowie die XSL-Datei, die alle Regeln zur Transformation enthält, mittels der Funktion \textit{transform:transform} umgewandelt. In diesem Fall entsteht eine Tabelle mit allen wichtigen Daten. eXist benutzt als XSLT-Prozessor standardmäßig \textit{Saxon HE}. Danach wird in den Zeilen 2 bis 15 ein XSL-FO-Dokument erstellt, welches Layout-Optionen für die PDF beinhaltet. Außerdem wird die transformierte XML als der Körper der PDF im Element \textit{$<$fo:block$>$} deklariert. In Zeile 16 entsteht durch die Funktion \textit{xslfo:render} das fertige PDF-Dokument. Dies kann nun an den Client als ein Base64 codierter String zurückgegeben und im Browser angezeigt werden. \textit{Apache FOP} ist in eXist der standardmäßige Renderer, der genutzt wird um die XSL-FO-Dokumente zu rendern. 
\\
\\
Da eXist aus einem NoSQL-DBMS besteht, werden alle Daten in Dokumenten abgespeichert. Dazu enthält die in diesem Projekt erstellte Datenbank für jede Person eine Collection (Siehe Abbildung \ref{collection1} auf S. \pageref{collection1}) mit der entsprechenden personId als Namen, um für die Reports gezielt nur die Dokumente zu durchsuchen, die notwendig sind. Innerhalb der Collections werden für jeden Tag separate XML-Dokumente (Siehe Abbildung \ref{collection2} auf S. \pageref{collection2}) erstellt, die alle Vitalwerte für die jeweilige Person enthalten.
\subsection{Vitalwerte Datengenerator}
Am Anfang war es wichtig, die vitale Werte d.H. Atemfrequenz, Herzfrequenz, Körpertemperatur und Blutdruck erzeugen lassen. Für die Darstellung kommt eine HTML Webseite mit Javascript und Jquery Technologien vor.

Um diese Aufgabe zu erreichen, wurde erst einfacher Formular erstellt.
Dieser besteht aus drei wichtigen Eingabefeldern, sprich Person ID, Room ID und Change room. Damit kann man eine fiktive Person mit einen Namen und einer Position darstellen. Um eine Bewegung zu simulieren, ist es später möglich das Zimmer zu ändern. Zusätzlich gibt es noch ein Start-und Stop Knopf für Ein-und Auschaltung des Generators.
 
Nachdem Einschalten werden die Werte für Atemfrequenz und Herzfrequenz jede 5 und für die Körpertemperatur und Blutdruck jede 10 Sekunden generiert. Die Werte werden inzwischen dem normalen Intervall erzeugt. Normaler Intervall enstspricht dem gesunden Menschen. Um den besseren Verständnis über die ganze Situation zu bekommen, besteht den Nutzern eine Möglichkeit die generierten vital Werte live zu beobachten und auch steuern. Der Nutzer kann die Werte erhöhen oder vermindern.

Sollten die Werte außerhalb des normalen Intervalles liegen, kommt es zur Warnung mit dem akustischen Signal. 
Der zweiten Element der Webseite ist eine Tabelle, die die neu generierten Werte zeigt. Dort sieht man die neue Werte, jetzige Person ID, Room ID und auch entsprechenden Zeitstempel. Es gibt auch s.g. Normalspalte, wo man der normalen Intervall sehen und auch mit neuen Werten vergleichen kann.
\\
\begin{lstlisting}
function generateBodyTemp() {
//from 35.8 to 37.3 Celsius
var x = parseFloat(Math.random() * 1.5 + 35.8).toFixed(2);
document.getElementById("bodyTemp").innerHTML = x;

var personId = document.getElementById("personid").value;
var roomId = document.getElementById("pokojid").innerHTML;
var time = displayTime();
var date = getDate();
var sensorType = "Koerpertemperatur"; 
var string = xmlString(personId, roomId, time, date, sensorType, x);

ajax(string);    
}
\end{lstlisting}

Zum Erläuterung des Generatorhintegrundes wurde eine Funktion generateBodyTemp genommen, die die fiktive Körpertemperatur erzeugt. Die restlichen Vitalwerte Funktionen basieren auf das gleichem Prinzip. Unser Ziel ist die Herstellung von Nummer in dem Intervall von 35.8 bis 37.3. In der dritten Zeile geht es um die zufällige Erzeugung des Wertes. Mit der Methode getNumberFromInterval kann man mittels dieser Formula die zufälligen Nummer in dem beliebigen Intervall herstellen. 
\begin{lstlisting}
var getNumberFromInterval = Math.random() * (max - min) + min;
\end{lstlisting}

Dazu wurde die Methode Math.random benutzt, die eine Nummer zwischen 0 und 1 liefert. Danach wird die Nummer mit multipliziert 1.5 und dazu noch 35.8 addiert. Mit der parseFloat Methode nimmt man den String und liefert eine Fließkommazahl zurück.  Die Methode toFixed konvertiert die Nummer in den String mit 2 Stellen nach dem Dezimalkomma. Die neue Nummer wird in die Variable x gespeichert. In der vierten Zeile wird den neu generierten Wert in der HTML Tabelle mittels getElementById Methode gezeigt.

In den Zeilen 6 und 7 werden die von dem User eingegebene Werte in den Variablen gespeichert. Die Zeilen Nr. 8 und 9 sind für die Herstellung des Zeitstempels zuständig. Dazu sind 2 Funktionen vorhanden und die Werte werden auch in den Variablen gespeichert. Die Zeile Nr. 10 beinhaltet den String Körpertemperatur. In der 11. Zeile wird die Funktion xmlString mit dem Parameter 6 benutzt. Diese Methode erhält die Parameter und speichert sie in die Variable string.

In der letzten Zeile wird die Funktion ajax mit dem Parameter string aufgerufen.
\\
\begin{lstlisting}
function ajax(string){
$.ajax({
url: 'http://uakk45c08dae.aalxmlprojekt.koding.io:8888/postVitalwert', 
processData: false,
type: "POST",
data: string,
crossDomain: true,
success: function(response){
console.log("Success!");
},
error: function(jqXHR, textStatus, errorThrown) {
console.log(errorThrown + ", " + jqXHR);
}
});
}
\end{lstlisting}

Die oben genannte Funktion ajax dient zum Übertragung der Daten zum Server. In der dritten Zeile spezifiziert man, zu welcher Adresse sich ajax verbindet. Die Variable string wird via POST Methode zu der URL gesendet. ProcessData ist auf false gesetzt, weil man das DOMDocument senden möchte. CrossDomain ist auf true gesetzt um die crossDomain Anforderungen verwenden zu können. Wenn die Daten erfolgreich zum Server gesendet werden, taucht die console mit dem Wert Success auf. Wenn aus irgendwelchen Gründen die Übertragung nicht ausgeführt werden kann, liefert die Funktion einen Error zurück.   
     
