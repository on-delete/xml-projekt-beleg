\section{Implementierung}

\subsection{NodeJS-Server}
Der Node.JS-Server besteht aus zwei Instanzen: Die eine Instanz ist als ein REST-Service implementiert. Sie stellt eine POST-Ressource zur Verfügung, über die die Daten der Sensoren an das eXist geschickt werden. Hierfür gibt es eine \textit{route}-Funktion, die ein Request entgegennimmt und prüft, welche URL angesprochen wird. 

\begin{lstlisting}
function route(handle, pathname, response, request) {
    if(typeof handle[pathname] === 'function'){
        handle[pathname](response, request);
    } else {
        console.log("No request handler found for " + pathname);
        response.writeHead(404, {"Content-Type": "text/plain"});
        response.write("404 Not found");
        response.end();
    }
}
\end{lstlisting}

Node.JS extrahiert den \textit{pathname}, also den Teil der URL, der nach dem Host und der Portnummer folgt, automatisch. Sollte die URL nicht zu einer Funktion zuordbar sein, wird ein 404-Fehler zurückgegeben.

\begin{lstlisting}
handle["/postVitalwert"] = vitalWerteHandler.postVitalwert;
\end{lstlisting}

Wenn aber die URL wie im obigen Beispiel \textit{postVitalwert} ist, wird die entsprechende Funktion \textit{postVitalwert} ausgeführt. Diese Funktion sendet die empfangenen Daten an eXist weiter. Um REST-Calls in Node.JS durchzuführen, wurde das Package \textit{node-rest-client} zusätzlich installiert. Ein POST kann wie folgt implementiert werden.

\begin{lstlisting}
client.post("http://localhost:8080/exist/restxq/postSensorData", args, function(data,response) {
    var string = data.message;
});
\end{lstlisting}

Dadurch wird ein POST an die angegebene Adresse mit den spezifischen Daten in \textit{args} ausgeführt. Die Antwort wird in der übergebenen Callback-Funktion verarbeitet. Außerdem  werden die Daten von den Sensoren an die zweite Instanz des Node.JS-Servers weitergeleitet.
\\
\\
Die zweite Instanz ist für die Kommunikation zwischen den Clients und dem Server verantwortlich. Diese wird durch den Einsatz von Websockets realisiert. Das Package \textit{socket.io} ermöglicht das Erzeugen einer Connection sowie das Senden und Empfangen von Events. Sobald sich ein Client über den Socket angemeldet hat, kann auf solche Events reagiert werden.

\begin{lstlisting}
socket.on("setPersonId", function(id){
            ...
});
\end{lstlisting}

In diesem Beispiel wird beim empfangenen Event \textit{setPersonId} die Callback-Funktion mit der empfangenen ID ausgeführt. Es können aber auch Events an die angemeldeten Clients mittels der Funktion \textit{emit} geschickt werden.

\begin{lstlisting}
receiver.emit("receiveVitalWert", xml);
\end{lstlisting}

Die beiden Instanzen wurden jeweils an zwei verschiedene Ports gebunden, um z.B. zu vermeiden, dass ein Websocket-Event an den Router des REST-Services geschickt wird. Deshalb wurde für den REST-Service der Port 8888 und für den Websocket-Service der Port 8887 festgelegt.

\subsection{eXist}
Um eXist von außerhalb ansprechen zu können, wurde der in eXist implementierte Standard \textit{RESTXQ} verwendet. Dieser ermöglicht mittels Annotationen das bereitstellen von \textit{resource functions}, da sie gewissermaßen Ressourcen über HTTP bereitstellen. Es können also XQuery-Funktionen über solche HTTP-Anfragen ausgeführt und die Ergebnisse zurückgegeben werden. POST-Methoden werden beispielsweise mit der Annotation \textit{\%rest:POST("\{\$body\}")} deklariert. Mit dem \textit{\$body} Element kann außerdem festgelegt werden, dass mit dem POST Daten gesendet werden sollen.

\begin{lstlisting}
declare
    %rest:POST("{$body}")
    %rest:consumes("application/xml")
    %rest:path("/postSensorData")
function ex:postSensorData($body){
    ...
};
\end{lstlisting}

Außerdem ist es möglich eine Antwort an den anfragenden Client innerhalb von XQuery zu erstellen. Das Element \textit{$<$rest:response$>$} erzeugt mit dem Kindelement \textit{$<$http:response$>$} eine HTTP Statusmeldung mit optional angehangenen Daten. Im folgenden Beispiel wird nach erfolgreichem Speichern der Vitalwerte ein Statuscode 200 gesendet mit dem kurzen XML-Dokument \textit{$<$message$>$Document updated$<$/message$>$}

\begin{lstlisting}
<rest:response>
     <http:response status="200" message="ok">
     <http:header name="Content-Type" value="application/xml"/>
     <http:header name="Access-Control-Allow-Origin" value="*"/>
</http:response>
</rest:response>,<message>Document updated</message>
\end{lstlisting}

Für die Erstellung der Reports wurde XSLT\footnote[1]{XSL Transformation} und XSL-FO\footnote[2]{Extensible Stylesheet Language – Formatting Objects} verwendet. 

\begin{lstlisting}
let $table-fo := transform:transform($tableBody,doc("/db/apps/aal-server/test-xsl.xsl"),())
let $fo := <fo:root xmlns:fo="http://www.w3.org/1999/XSL/Format">
    <fo:layout-master-set>
        <fo:simple-page-master master-name="my-page">
            <fo:region-body margin="1in"/>
        </fo:simple-page-master>
    </fo:layout-master-set>
    <fo:page-sequence master-reference="my-page">
        <fo:flow flow-name="xsl-region-body">
            <fo:block>
                {$table-fo}
            </fo:block>
        </fo:flow>
    </fo:page-sequence>
</fo:root>
let $pdf := xslfo:render($fo, "application/pdf", ())
\end{lstlisting}

In der ersten Zeile wird das XML-Dokument, welches transformiert werden soll sowie die XSL-Datei, die alle Regeln zur Transformation enthält, mittels der Funktion \textit{transform:transform} umgewandelt. In diesem Fall entsteht eine Tabelle mit allen wichtigen Daten. eXist benutzt als XSLT-Prozessor standardmäßig \textit{Saxon HE}. Danach wird in den Zeilen 2 bis 15 ein XSL-FO-Dokument erstellt, welches Layout-Optionen für die PDF beinhaltet. Außerdem wird die transformierte XML als der Körper der PDF im Element \textit{$<$fo:block$>$} deklariert. In Zeile 16 entsteht durch die Funktion \textit{xslfo:render} das fertige PDF-Dokument. Dies kann nun an den Client als ein Base64 codierter String zurückgegeben und im Browser angezeigt werden. \textit{Apache FOP} ist in eXist der standardmäßige Renderer, der genutzt wird um die XSL-FO-Dokumente zu Rendern. 
\\
\\
Da eXist aus einem NoSQL-DBMS besteht, werden alle Daten in Dokumenten abgespeichert. Dazu enthält die in diesem Projekt erstellte Datenbank für jede Person eine Collection (Siehe Abbildung \ref{collection1} auf S. \pageref{collection1}) mit der entsprechenden personId als Namen, um für die Reports gezielt nur die Dokumente zu durchsuchen, die notwendig sind. Innerhalb der Collections werden für jeden Tag separate XML-Dokumente (Siehe Abbildung \ref{collection2} auf S. \pageref{collection2}) erstellt, die alle Vitalwerte für die jeweilige Person enthalten.
\subsection{Datengenerator}
Am Anfang war es wichtig, die vitale Werte d.H. Atemfrequenz, Herzfrequenz, Körpertemperatur und Blutdruck erzeugen lassen. Für die Darstellung kommt eine HTML Webseite mit Javascript und Jquery Technologien vor.
\\
\\
Um diese Aufgabe zu erreichen, wurde erst einfacher Formular erstellt.
Dieser besteht aus drei wichtigen Eingabefeldern, sprich Person ID, Room ID und Change room. Dazu gibt es noch ein Start und Stop Knopf für Ein-und Auschaltung des Generators. 
Nachdem Einschalten werden die Werte für Atemfrequenz und Herzfrequenz jede 5 und für die Körpertemperatur und Blutdruck jede 10 Sekunden generiert. Die Werte werden inzwischen dem normalen Intervall erzeugt. Normaler Intervall enstspricht dem gesunden Menschen.
\\
\\
Um den besseren Verständnis über die ganze Situation zu bekommen, besteht den Nutzern eine Möglichkeit die generierten vital Werte live zu beobachten und auch steuern. Der Nutzer kann die Werte erhöhen oder vermindern. Abhängig von der Werten kann auch die Warnung mit dem akustischen Signal auftauchen. 
Der zweiten Element der Webseite ist eine Tabelle, die die neu generierten Werte zeigt. Dort sieht man die generierten Werte, jetzige Person ID, Room ID und auch entsprechenden Zeitstempel. Es gibt auch s.g. OK Intervall Spalte, wo man der normalen Intervall sehen und auch mit neuen Werten vergleichen kann.