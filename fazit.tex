\section{Fazit}

\subsection{Probleme}
Während der Bearbeitung des Projektes kam es zu diversen Problemen, die sich aber hauptsächlich auf den Funktionsumfang und dem Implementierungsstand des sich noch in der Entwicklung befindenden eXist konzentrierten. Um Websockets innerhalb von eXist nutzen zu können, war zunächst angedacht, das Package XQWebsocketModule zu verwenden, da dies eine Implementierung der Sockets als ein Javaprogramm und die Verwendung innerhalb von XQuery versprach. Jedoch konnte für das Package keine gültige Archivdatei erstellt werden, welche von eXist gelesen und importiert werden kann. Ein möglicher Grund dafür könnte sein, dass dieses Modul nicht für die aktuelle Version 2.2 entwickelt wurde. Laut Github, wo das Projekt gehostet wurde, gab es seit rund zwei Jahren keine größeren Änderungen mehr. Deswegen wurde letztendlich entschieden, das Modul nicht zu verwenden, sondern die Kommunikation per Websockets auf einen Node.JS-Server auszulagern.
\\
\\
Weiterhin gab es bei der Implementierung der Queries mittels XQuery diverse Probleme bei der Benutzung bestimmter Funktionen. Beispielsweise war es mit der Funktion \textit{file:exists}, nicht möglich zu bestimmen, ob eine Datei bereits existiert oder nicht. Im konkreten Fall gab die Funktion immer \textit{False} zurück, obwohl die Datei vorhanden war. Vermutlich lässt sich dieses Problem wieder auf die nicht abgeschlossene Entwicklung von eXist zurückführen, sodass noch nicht alle von der W3C\footnote[1]{World Wide Web Consortium} definierten Funktionen in XQuery 3.0 implementiert zu sein scheinen. Gelöst wurde dieses Problem durch die Verwendung der Funktion \textit{doc-available} aus der XQuery Version 1.0, die ein zufriedenstellendes Ergebnis lieferte.

\subsection{Ausblick}